% \documentclass[twoside,9pt]{extarticle}
    \documentclass[12pt,twoside]{article}
        \usepackage{jmlda}
        
        \usepackage{listings}
        \usepackage{caption}
        %\DeclareCaptionFont{white}{\color{white}}
        %\DeclareCaptionFormat{listing}{\colorbox{gray}{\parbox{\textwidth}{#1#2#3}}}
        %\captionsetup[lstlisting]{format=listing,labelfont=white,textfont=white}
        %%\NOREVIEWERNOTES
        
        % \newenvironment{coderes}%
        %     {\medskip\tabcolsep=0pt\begin{tabular}{>{\small}l@{\quad}|@{\quad}l}}%
        %     {\end{tabular}\medskip}
        
        \begin{document}
        \English
        
        \title{On conformational changes of proteins using collective motions in torsion angle space and L1 regularization}
        \author{Ryabinina~R.\,B., Emtsev~D.\,I.}
            [Ryabinina~R.\,B.$^1$, 
            Emtsev~D.\,I.$^2$ ]
        \email{$^1$ ryabinina.rb@phystech.edu 
               $^2$ daniil.emcev.ru@yandex.ru 
               }
        \organization{$^{1\,2}$MIPT}
        
        \abstract{
        \titleEng{Abstract}
        \abstractEng{
            Torsion angles are the most natural degrees of freedom
             for describing motions of polymers, such as proteins.
              This is because bond lengths and bond angles are heavily
            constrained by covalent forces. Thus, multiple attempts
            have been done to describe protein dynamics in the torsion angle space...
               The goal of the current project is to study if a sparse subset of collective coordinates in the torsion subspace can describe functional conformational changes in proteins. 
               This will require a solution of a ridge regression problem with a L1 regularization constraint. The starting point will be the LASSO formulation...
            }
        }
        
        \maketitle
        
        
        \end{document}